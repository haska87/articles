\documentclass[runningheads,a4paper]{llncs}

\usepackage{amssymb}
\setcounter{tocdepth}{3}
\usepackage{graphicx}
\graphicspath{{./figs/}}

\usepackage{amsmath} 
\usepackage{bm}
\usepackage{amsbsy} 
\usepackage{amssymb}
\usepackage{cite}

\usepackage{color, colortbl}
\definecolor{col1}{gray}{0.8}
\definecolor{col2}{gray}{0.9}
\definecolor{col3}{gray}{0.95}


\newcommand{\keywords}[1]{\par\addvspace\baselineskip
\noindent\keywordname\enspace\ignorespaces#1}

\newcommand{\grad}{\mathop{\rm grad}\nolimits}
\newcommand{\const}{\mathop{\rm const}\nolimits}
\renewcommand{\div}{\mathop{\rm div}\nolimits}
\newcommand{\half}{\frac{1}{2}}

\title{Solution of the 3D Neutron Diffusion Benchmark by FEM}

\author{A.V. Avvakumov$^{1}$, P.N. Vabishchevich$^{2}$, A.O. Vasilev$^{3,a)}$ and V.F. Strizhov$^{2}$}

\institute{$^1$National Research Center \emph{Kurchatov Institute}, Moscow, Russia \\
$^2$Nuclear Safety Institute of RAS,  Moscow, Russia \\
$^3$North-Eastern Federal University, Yakutsk, Russia\\
$^{a)}$Corresponding author: \url{haska87@gmail.com}}

\titlerunning{Solution of the 3D Neutron Diffusion Benchmark by FEM}
\begin{document}

\maketitle

\begin{abstract}
The objective is to analyze the neutron diffusion benchmark developed by Atomic Energy Research for the verification of best-estimate neutronics codes. The 3D benchmark of Schulz models a VVER-1000 core in steady state. The assemblies are homogeneous, represented by given two-group diffusion theory parameters. There are seven material compositions including four enrichments, burnable absorber, control rods and reflector. The finite element method on triangular calculation grids is used to solve the three-dimensional neutron problem. The software has been developed using the engineering and scientific calculation library FEniCS. The matrix spectral problem is solved using a scalable and flexible toolkit SLEPc. The solution accuracy of the benchmark is analyzed by condensing calculation grid and varying degree of finite elements.
\end{abstract}

%\keywords{Neutron diffusion equation, multi-group approximation, space-time kinetic, spectral problem.}


\section{Introduction}
The physical processes in a nuclear reactor \cite{duderstadt1976nuclear}
depend on distribution of neutron flux, whose mathematical description is based on the neutron-transport equation \cite{hetrick1971dynamics,stacey2007}. 
The general view of this equation is integrally-differential one, and the required distribution of neutrons flux depends on time, energy, spatial and angular variables. As a rule, the simplified forms of the neutron transport equation are used for practical calculations of nuclear reactors. The equation system that is known as a multigroup diffusion approach is mostly used for reactor analysis \cite{marchuk1986numerical,lewis1993computational} and is applied in most engineering calculation codes.
%sutton1996diffusion,cho2005fundamentals

The processes in a nuclear reactor are essentially non-stationary. The stationary state of neutron flux, which is related to the critical state of the reactor, is characterised by local balancing of neutron absorption and birth intensities. This boundary state is usually described by solution of a spectral problem (Lambda Modes problem, $\lambda$-eigenvalue problem) provided that the fundamental eigenvalue (maximal eigenvalue) that is called k-effective of the reactor core, is equal to unity. In this case, the stationary neutron field is a corresponding eigenfunction. Calculations of k-effective of the reactor on the basis of the spectral Lambda Modes problem solution are obligatory for developing a new design of reactor installation.

\section{Problem statement}

Let's consider modelling neutron flux in a multi-group diffusion approximation. Neutron flux dynamics is considered within limited 2D or 3D domain  $\Omega$ ($\bm x = \{x_1, ..., x_d\} \in \Omega, \ d = 2,3$) with a convex boundary $\partial \Omega$. Neutron transport is described by the set of equations without taking into account delayed neutron source:
 \begin{equation}\label{1}
\begin{split}
 \frac{1}{v_g} \frac{\partial \phi_g}{\partial t} - & \nabla \cdot D_g \nabla \phi_g + \Sigma_{rg} \phi_g 
 - \sum_{g\neq g'=1}^{G} \Sigma_{s,g'\rightarrow g} \phi_{g'} \\
 =  & \ ( (1-\beta) \chi_g + \beta \widetilde{\chi}_g) \sum_{g'=1}^{G} \nu \Sigma_{fg'} \phi_{g'} , \quad 
 g = 1,2, ..., G .
\end{split}
\end{equation} 
Here $\phi_g(\bm x,t)$ --- neutron flux of $g$ group at point $\bm x$ and time $t$,
$G$ --- number of groups,
$v_g$ --- effective velocity of neutrons in the group $g$,
$D_g(\bm x)$ --- diffusion coefficient, $\Sigma_{rg}(\bm x,t)$ --- removal cross-section,
$\Sigma_{s,g'\rightarrow g}(\bm x,t)$ --- scattering cross-section from group $g'$ to group $g$,
$\beta$ --- effective fraction of delayed neutrons, $\chi_g$, $\widetilde{\chi}_g$  --- spectra of instantaneous and delayed neutrons, 
$\nu\Sigma_{fg}(\bm x,t)$ --- generation cross-section of group $g$.
The albedo-type conditions are set at the boundary $\partial \Omega$:
\begin{equation}\label{2}
 D_g\frac{\partial \phi_g}{\partial n} + \gamma_g \phi_g = 0, \quad 
 \quad g = 1,2, ..., G ,
\end{equation}
where $n$ is the outer normal to the boundary $\partial \Omega$.
Let's consider the problem for equation (\ref{1}) with boundary conditions (\ref{2}), and initial conditions:
\begin{equation}\label{3}
 \phi_g(\bm x,0) = \phi_g^0(\bm x), 
  \quad  g = 1,2, ..., G .
\end{equation} 

Let's write the boundary problem (\ref{1}), (\ref{2}), (\ref{3}) in operator notation. 
We define the vector $\bm \phi = \{\phi_1, \phi_2, ..., \phi_G\}$ and the matrices
\[
 V = (v_{g g'}),
 \quad v_{g g'} = \delta_{g g'} v_g^{-1},
\] 
\[
 D = (d_{g g'}),
 \quad d_{g g'} = - \delta_{g g'} \nabla \cdot D_g \nabla,
\] 
\[
 S = (s_{g g'}),
 \quad  s_{g g'} =  \delta_{g g'} \Sigma_{rg} - \Sigma_{s,g'\rightarrow g} ,
\] 
\[
 R = (r_{g g'}),
 \quad  r_{g g'} = ( (1-\beta) \chi_g + \beta \widetilde{\chi}_g) \nu \Sigma_{fg'} ,
\]
\[
g, g' = 1,2, ..., G,
\] 
where
\[
 \delta_{g g'} = \left \{ 
 \begin{matrix}
 1, & g = g', \\
 0, & g \neq  g',
 \end{matrix}
 \right . 
\] 
is the Kronecker delta. We shall use the set of vectors $\bm \phi$,  whose components satisfy the boundary conditions
(\ref{3}). Using introduced definitions, the system of equations (\ref{1}) can be written in the form of the first-order evolutionary equation:
\begin{equation}\label{4}
 V \frac{d \bm \phi}{d t} + (D+S) \bm \phi = R \bm \phi .
\end{equation}  
The Cauchy problem is solved for (\ref{4}), when
\begin{equation}\label{5}
 \bm \phi(0) = \bm \phi^0,
\end{equation} 
where $\bm \phi^0 = \{ \phi_1^0,  \phi_2^0, ...,  \phi_G^0 \}$.

To characterize the dynamic processes in a nuclear reactor, which are described by Cauchy problem (\ref{4}), (\ref{5}), solutions of some spectral problems  \cite{publicAnnnals2017}. Let's consider the solution of the spectral problem, called Lambda Modes problem:
\begin{equation}\label{6}
 (D+S) \bm \varphi  = \lambda^{(k)} R \bm \varphi .
\end{equation} 
This problem (\ref{6}) is known as the Lambda modes problem for a given configuration of the reactor core.
The minimal eigenvalue is used for characterisation of neutron field, thus 
\[
 k = \frac{1}{\lambda^{(k)}_1}  
\] 
is the effective multiplication factor.

\section{AER Benchmark}
The 3D benchmark of Schulz \cite{schulz1996} models a VVER-1000 core in steady state. The assemblies are homogeneous, represented by given two-group diffusion theory parameters. There are seven material compositions including four enrichments, burnable absorber, control rods and reflector. The core height is 355 cm, covered with axial and radial reflectors. The nodes are large, with assembly lattice pitch of 24.1 cm. Figure \ref{fig:1} shows the radial and axial geometry models.  Diffusion neutronics constants in the common notations
are given in Table \ref{t-1}. 

%\begin{figure}[htp]
%  \begin{center}
%    \includegraphics[width=0.45\linewidth] {1.png}
%	\caption{Radial geometry model}
%	\label{fig:1}
%  \end{center}
%\end{figure} 
%
%\begin{figure}[htp]
%  \begin{center}
%    \includegraphics[width=0.45\linewidth] {2.png}
%	\caption{Axial geometry model}
%	\label{fig:2}
%  \end{center}
%\end{figure}

\begin{figure}[h]
\begin{center}
\begin{minipage}{0.48\linewidth}
\center{\includegraphics[width=1\linewidth] {1.png}}\\
\end{minipage}
\hfill
\begin{minipage}{0.48\linewidth}
\center{\includegraphics[width=1\linewidth] {2.png}}\\
\end{minipage}
\hfill
\caption{Radial and axial geometry model.}
\label{fig:1}
  \end{center}
\end{figure}


\begin{table}[htp]
\caption{Diffusion neutronics constants.}
\label{t-1}
\begin{center}
\begin{tabular}{ccrrrrr}
\rowcolor{col1}
Material & Group & \multicolumn{1}{c}{$D$} & \multicolumn{1}{c}{$\Sigma_r$} & \multicolumn{1}{c}{$\Sigma_{1\to 2}$} & \multicolumn{1}{c}{$\Sigma_f$}& \multicolumn{1}{c}{$\nu\Sigma_f$}\\
\rowcolor{col3}
1 & 1 & 1.37548 & ~2.4135e-2 & ~1.5946e-2 & ~6.0130e-7 & ~4.7663e-3 \\
\rowcolor{col2}
  & 2 & 0.38333 & 6.6002e-2 &           & 1.1231e-5 & 8.3980e-2 \\
\rowcolor{col3}
2 & 1 & 1.40950 & 2.4769e-2 & 1.4346e-2 & 5.9305e-7 & 4.7020e-3 \\
\rowcolor{col2}
  & 2 & 0.38756 & 7.4988e-2 &           & 1.1253e-5 & 8.4128e-2 \\
\rowcolor{col3}
3 & 1 & 1.37067 & 2.3800e-2 & 1.5172e-2 & 7.4429e-7 & 5.8437e-3 \\
\rowcolor{col2}
  & 2 & 0.38028 & 8.0442e-2 &           & 1.5336e-5 & 1.1468e-1 \\
\rowcolor{col3}
4 & 1 & 1.39447 & 2.4069e-2 & 1.3903e-2 & 7.8731e-7 & 6.1632e-3 \\
\rowcolor{col2}
  & 2 & 0.38549 & 9.4773e-2 &           & 1.6848e-5 & 1.2598e-1 \\
\rowcolor{col3}
5 & 1 & 1.36938 & 2.3697e-2 & 1.4855e-2 & 8.1014e-7 & 6.3396e-3 \\
\rowcolor{col2}
  & 2 & 0.37877 & 8.7681e-2 &           & 1.7381e-5 & 1.2998e-1 \\
\rowcolor{col3}
6 & 1 & 1.00000 & 4.0644e-2 & 2.4875e-2 & 0.0000e-0 & 0.0000e-0 \\
\rowcolor{col2}
  & 2 & 0.33333 & 5.2785e-2 &           & 0.0000e-0 & 0.0000e-0 \\
\rowcolor{col3}
7 & 1 & 1.36966 & 2.3721e-2 & 1.4927e-2 & 7.9536e-7 & 6.2284e-3 \\
\rowcolor{col2}
  & 2 & 0.37911 & 8.5850e-2 &           & 1.6866e-5 & 1.2612e-1 \\

\end{tabular}
\end{center}
\end{table}

\section{Computional results}
In the framework of used two-group model, the spectral problem (\ref{6}) can be written as:
\begin{equation}\label{7}
\begin{split}
 - \nabla \cdot D_1 \nabla \varphi_1 & + \Sigma_{r1} \varphi_1  
 = \lambda^{(k)} (\nu \Sigma_{f1} \varphi_1 + \nu \Sigma_{f2} \varphi_2), \\
 - \nabla \cdot D_2 \nabla \varphi_2 & + \Sigma_{r2} \varphi_2 - \Sigma_{s,1\rightarrow 2} \varphi_1  
 = 0.
\end{split}
\end{equation}
The boundary conditions (\ref{2}) are used at $\gamma_g = 0.5, \ g = 1,2$.

To obtain approximate solutions we use the finite element methods \cite{brenner} 
with a tetrahedron calculational mesh. The software has been developed using the engineering and scientific calculation library FEniCS \cite{logg2012automated}. The SLEPc package \cite{slepc} has been used for numerical solution of the spectral problems:
\begin{equation}\label{8}
A\bm{x} = \lambda B \bm{x}.
\end{equation}
Used Krylov-Schur algorithm with an accuracy of $10^{-15}$. In the calculations the following parameters were varied:
\begin{itemize}\itemsep1pt \parskip0pt \parsep0pt
\item $\kappa$ is the number of tetrahedrons per one assembly; 
\item $z$ is the number of tetrahedrons in height; 
\item $p$ is order of finite element.
\end{itemize}
The number of tetrahedrons per one assembly $\kappa$  varies from 6 to 96 and $z$ varies from 12 to 48. In Figure \ref{fig:2} shows discretization of assembly at $\kappa=6$ and $z=12$, $\kappa=24$ and $z=24$, $\kappa=96$ and $z=48$, respectively. The standard Lagrangian finite elements of degree $p=1,2,3$ are used.

\begin{figure}[htp]
  \begin{center}
    \includegraphics[width=0.65\linewidth] {3.png}
	\caption{Different types of discretization of assembly.}
	\label{fig:2}
  \end{center}
\end{figure}

The following parameters were calculated:
\begin{itemize}\itemsep1pt \parskip0pt \parsep0pt
\item $k$ is effective multiplication factor;
\item $P$ is power distribution  per assembly with the normalization of the mean value of the core:
\begin{equation}\label{9}
P = a(\Sigma_{f1} \varphi_1 + \Sigma_{f2} \varphi_2),
\end{equation}
where $a$ is normalization coefficient by a given value of the integral power.
\end{itemize}

Comparison of the results was carried out with the results of calculations performed by diffusion program CRONOS \cite{cronos}. The extrapolated finite-element solution of of the second-order CRONOS results is recommended as the reference solution ($k_{ref} = 1.049526$) of the Schulz benchmark. 

%CRONOS is a reactor code of CEA which uses finite elements and nodal methods for homogenized diffusion and transport calculations.
%The deviations between the CRONOS finest and extrapolated 3D solution, which characterize the accuracy of the recommended solution are given in Table 3.

Let's consider the following variations in the calculated parameters:
\begin{itemize}\itemsep1pt \parskip0pt \parsep0pt
\item for the effective multiplication factor, absolute deviation from the reference value $k_{ref}$: $\Delta k = |k - k_{ref}|$, expressed in \textit{pcm} (percent-milli, i.e. $10^{-5}$);
\item for power distribution per assembly $P_i$ calculated relative deviation $\varepsilon_i$ (expressed in \%):
\[
\varepsilon_i = \frac{P_i - P_i^{ref}}{P_i^{ref}},
\]
where $P_i^{ref}$ --- «reference» value of power per assembly $i$ ($i = 1,...,N_e$).
\item by deviations $\varepsilon_i$ calculated integral deviation:
\begin{itemize}\itemsep1pt \parskip0pt \parsep0pt
\item the root mean deviation RMS:
\[
\mathrm{RMS} = \sqrt{\frac{1}{N_e}\sum_{i=1}^{N_e} \varepsilon_i^2},
\]
\item the mean absolute deviation AVR:
\[
\mathrm{AVR} = \frac{1}{N_e}\sum_{i=1}^{N_e} \left\vert \varepsilon_i\right\vert,
\]
\item the maximum modulus deviation MAX:
\[
\mathrm{MAX} = \underset{i}{\max}\left\vert\varepsilon_i\right\vert.
\]
\end{itemize}
\end{itemize}

Results of the solution of $\lambda$-spectral problem (\ref{6}) for the main eigenvalue $k_1 $ using different grids and finite elements are given in Table. \ref{t-2}. Here $N$ is  the size of the matrices $A,B$ (\ref{8}). These data demonstrate the convergence of the computed eigenvalues with thickening of the grid $\kappa$, $z$ and with increasing the degree $p$. Comparison of the reference solutions with the result of the solution at the parameters $p=2, \kappa=6, z=24$ (in table \ref{t-2} highlighted in green) is shown in Figure \ref{fig:3}.


Table \ref{t-3} shows the results obtained for the next three eigenvalues for different meshes.
Vertical and horizontal cuts of power distribution for the first four eigenvalues are shown in Figure \ref{fig:4}.

\begin{table}[htp]
\caption{$k_1$ results for the Schulz benchmark.}
\label{t-2}
\begin{center}
\begin{tabular}{rrrrrrrrr}
\rowcolor{col1}
$p$ & $\kappa$ & $z$ &\multicolumn{1}{c}{$k_1$} & \multicolumn{1}{c}{$\Delta k$} & \multicolumn{1}{c}{MAX} & \multicolumn{1}{c}{AVR}& \multicolumn{1}{c}{RMS}& N \\
\rowcolor{col3}
~1& ~~6& ~12& ~1.0476057& ~192.03& ~7.9382& ~2.5145& ~2.7918& 18,278 \\
\rowcolor{col2}
1& 6& 24& 1.0484070& 111.90& 7.6614& 2.3465& 2.5983& 35,150 \\
\rowcolor{col1}
1& 6& 48& 1.0486511& 87.49& 7.6793& 2.3643& 2.6092& 68,894 \\
\rowcolor{col3}
1& 24& 12& 1.0482940& 123.20& 2.2234& 0.5665& 0.7041& 70,382 \\
\rowcolor{col3}
1& 24& 24& 1.0487937& 73.23& 1.9377& 0.4050& 0.4774& 135,350 \\
\rowcolor{col1}
1& 24& 48& 1.0493645& 16.15& 1.9823& 0.3980& 0.4612& 265,286\\
\rowcolor{col3}
1& 96& 12& 1.0483122& 121.38& 1.2015& 0.3780& 0.4857& 276,146\\
\rowcolor{col2}
1& 96& 24& 1.0490651& 46.09& 0.2647& 0.1129& 0.1389& 531,050\\
\rowcolor{col1}
1& 96& 48& 1.0493997& 12.63& 0.4554& 0.1019& 0.1243& 1,040,858\\
\rowcolor{col3}
2& 6& 12& 1.0496463& -12.03& 0.9739& 0.4801& 0.5581& 135,350\\
\rowcolor{green}
2& 6& 24& 1.0497290& -20.30& 0.9576& 0.4504& 0.5314& 265,286\\
\rowcolor{col1}
2& 6& 48& 1.0497379& -21.19& 0.9576& 0.4501& 0.5307& 525,158\\
\rowcolor{col3}
2& 24& 12& 1.0494978& 2.82& 0.3246& 0.1577& 0.1861& 531,050\\
\rowcolor{col2}
2& 24& 24& 1.0495665& -4.05& 0.2597& 0.1176& 0.1414& 1,040,858\\
\rowcolor{col1}
2& 24& 48& 1.0495858& -5.98& 0.2435& 0.1117& 0.1335& 2,060,474\\
\rowcolor{col3}
2& 96& 12& 1.0494551& 7.09& 0.1786& 0.0662& 0.0771& 2,103,650\\
\rowcolor{col2}
2& 96& 24& 1.0495265& -0.05& 0.0844& 0.0309& 0.0377& 4,123,154\\
\rowcolor{col1}
2& 96& 48& 1.0495471& -2.11& 0.0573& 0.0256& 0.0298& 8,162,162\\
\rowcolor{col3}
3& 6& 12& 1.0495750& -4.90& 0.2149& 0.0956& 0.1153& 444,962 \\
\rowcolor{col2}
3& 6& 24& 1.0495782& -5.22& 0.2110& 0.0931& 0.1125& 877,898  \\
\rowcolor{col1}
3& 6& 48& 1.0495771& -5.11& 0.2005& 0.0900& 0.1082& 1,743,770\\
\rowcolor{col3}
3& 24& 12& 1.0495406& -1.46& 0.0430& 0.0177& 0.0216& 1,756,982\\
\rowcolor{col2}
3& 24& 24& 1.0495382& -1.22& 0.0317& 0.0123& 0.0146& 3,466,478\\
\rowcolor{col1}
3& 24& 48& 1.0495381& -1.21& 0.0286& 0.0102& 0.0126& 6,885,470\\
\rowcolor{col3}
3& 96& 12& 1.0495357& -0.97& 0.0317& 0.0106& 0.0137& 6,982,418\\
\rowcolor{col2}
3& 96& 24& 1.0495338& -0.78& 0.0211& 0.0092& 0.0110& 13,776,122\\
\rowcolor{col1}
3& 96& 48& 1.0495336& -0.76& 0.0162& 0.0080& 0.0100& ~27,363,530\\
\end{tabular}
\end{center}
\end{table}

\begin{figure}[htp]
  \begin{center}
    \includegraphics[width=1\linewidth] {power.png}
	\caption{Power distribution at $p=2, \kappa=6, z=24$.}
	\label{fig:3}
  \end{center}
\end{figure}

\begin{table}[htp]
\caption{The eigenvalues $k_2$, $k_3$ и $k_4$.}
\label{t-3}
\begin{center}
\begin{tabular}{rrrrrrr}
\rowcolor{col1}
$p$ & $\kappa$ & $z$ &\multicolumn{1}{c}{$k_2$} & \multicolumn{1}{c}{$k_3$} & \multicolumn{1}{c}{$k_4$} \\
\rowcolor{col3}
~1& ~~6& ~12& ~1.0367578& ~1.0367505& ~1.0265919 \\
\rowcolor{col2}
1& 6& 24& 1.0378381&	1.0378355&	1.0286801 \\
\rowcolor{col1}
1& 6& 48& 1.0381457&	1.0381434&	1.0292950\\
\rowcolor{col3}
1& 24& 12& 1.0378816&	1.0378762&	1.0277542\\
\rowcolor{col3}
1& 24& 24& 1.0389665&	1.0389631&	1.0299569\\
\rowcolor{col1}
1& 24& 48& 1.0392872&	1.0392857&	1.0306072\\
\rowcolor{col3}
1& 96& 12& 1.0378388&	1.0378365&	1.0276575\\
\rowcolor{col2}
1& 96& 24& 1.0390329&	1.0390303&	1.0300412\\
\rowcolor{col1}
1& 96& 48& 1.0393812&	1.0393798&	1.0307428\\
\rowcolor{col3}
\end{tabular}
\end{center}
\end{table}

\begin{figure}[htp]
  \begin{center}
\begin{minipage}{0.2\linewidth}
\center{\includegraphics[width=1\linewidth]{u1v.png}}\\
vertical for $k_1$
\end{minipage}
\hfill
\begin{minipage}{0.2\linewidth}
\center{\includegraphics[width=1\linewidth]{u1h.png}}\\
horizontal for $k_1$
\end{minipage}
\hfill
\begin{minipage}{0.2\linewidth}
\center{\includegraphics[width=1\linewidth]{u2v.png}}\\
vertical for $k_2$
\end{minipage}
\hfill
\begin{minipage}{0.2\linewidth}
\center{\includegraphics[width=1\linewidth]{u2h.png}}\\
horizontal for $k_2$
\end{minipage}
\vfill
\begin{minipage}{0.2\linewidth}
\center{\includegraphics[width=1\linewidth]{u3v.png}}\\
vertical for $k_3$
\end{minipage}
\hfill
\begin{minipage}{0.2\linewidth}
\center{\includegraphics[width=1\linewidth]{u3h.png}}\\
horizontal for $k_3$
\end{minipage}
\hfill
\begin{minipage}{0.2\linewidth}
\center{\includegraphics[width=1\linewidth]{u4v.png}}\\
vertical for $k_4$
\end{minipage}
\hfill
\begin{minipage}{0.2\linewidth}
\center{\includegraphics[width=1\linewidth]{u4h.png}}\\
horizontal for $k_4$
\end{minipage}

\caption{Sections.}
\label{fig:4}
  \end{center}
\end{figure}

\section{Acknowledgements}
The research was supported by the Government of the Russian Federation (project 14.Y26.31.0013).

\begin{thebibliography}{12}
\bibitem{duderstadt1976nuclear}
Duderstadt, J.J., Hamilton, L.J.: Nuclear Reactor Analysis. Wiley (1976)

\bibitem{stacey2007}
Stacey, W.M.: Nuclear Reactor Physics. Wiley (2007)

\bibitem{hetrick1971dynamics}
Hetrick, D.L.: Dynamics of Nuclear Reactors. University of Chicago Press (1971)

\bibitem{marchuk1986numerical}
Marchuk, G.I., Lebedev, V.I.: Numerical Methods in the Theory of Neutron Transport. Harwood Academic Pub  (1986)

\bibitem{lewis1993computational}
Lewis, E.E., Miller, W.F.: Computational Methods of Neutron Transport. American Nuclear Society (1993)

\bibitem{publicAnnnals2017}
Avvakumov, A.V., Vabishchevich, P.N., Vasilev, A.O., Strizhov, V.F.: Spectral properties of dynamic processes in a nuclear reactor. Annals of Nuclear Energy, vol. 99, pp. 68--79 (2017)

\bibitem{schulz1996}
Schulz, G.: Solution of a 3D VVER-1000 benchmark. In Proc. of 6-th Symposium of AER, Kikkonummi, Finland (1996)

\bibitem{brenner}
Brenner, S.C., Scott, R.: The Mathematical Theory of Finite Element Methods. Springer (2008)

\bibitem{logg2012automated}
Logg, A., Mardal, K.A., Wells, G.: Automated solution of differential equations by the finite element method: The FEniCS book. Springer Science \& Business Media, vol. 84 (2012)

\bibitem{slepc}
Campos, C., Roman, J.E., Romero, E., Tomas, A.: SLEPc Users Manual, \url{http://slepc.upv.es/documentation/manual.htm} (2013)

\bibitem{cronos}
Kolev, N.P., Lenain, R., Fedon-Magnaud, C.: Solutions of the AER 3D Benchmark for VVER-1000 by CRONOS”. Proc. 7-th Symposium of AER on VVER Reactor Physics and Safety, Hoernitz, Germany (1997)
  
\end{thebibliography}


\end{document}
