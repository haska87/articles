\documentclass[runningheads]{llncs}

\usepackage{graphicx}
\graphicspath{{./figs/}}

\usepackage{amsmath} 
\usepackage{bm}
\usepackage{amsbsy} 
\usepackage{amssymb}
\usepackage{cite}

\begin{document}
%
\title{Multiscale model reduction for neutron diffusion equation}
%
%\titlerunning{Abbreviated paper title}
% If the paper title is too long for the running head, you can set
% an abbreviated paper title here
%
\author{A.O. Vasilev\inst{1} \and 
D.A. Spiridonov\inst{1} \and
A.V. Avvakumov\inst{2} }
%
\authorrunning{A.O. Vasilev, D.A. Spiridonov \and A.V. Avvakumov}
% First names are abbreviated in the running head.
% If there are more than two authors, 'et al.' is used.
%
\institute{North-Eastern Federal University, Yakutsk, Russia \and
National Research Center \emph{Kurchatov Institute}, Moscow, Russia\\
\email{haska87@gmail.com}
}
%
\maketitle              % typeset the header of the contribution
%
\begin{abstract}
Construction of multiscale method for neutron diffusion equation is provided. 
The neutron diffusion approximation is mostly used for reactor analysis and applied in most engineering calculation codes. 
The solution method is based on the use of a generalized multiscale finite element method.
The main idea is to create local solutions that can be used to effectively solve on a coarse grid.
Numerical results show that multiscale basis functions can well take into account the small-scale characteristics of the medium and provide accurate solutions. 

\keywords{multiscale method \and parabolic equation \and neutron diffusion}
\end{abstract}

\section{Introduction}
bla bla bla

\section{Problem statement}
Let's consider modelling neutron flux in a one-group diffusion approximation. Neutron flux dynamics is considered within a bounded 2D domain  $\Omega$ ($\bm x = \{x_1, x_2\} \in \Omega$) with a convex boundary $\partial \Omega$. The neutron diffusion is described by the following set of equations with one group delayed neutron source:
\begin{equation}\label{1}
\begin{split}
 \frac{1}{v} \frac{\partial \phi}{\partial t} - \nabla \cdot D \nabla \phi + \Sigma_r \phi &= \frac{1 - \beta}{K_{eff}} \nu \Sigma_f \phi + \lambda c, \\
\frac{\partial c}{\partial t} + \lambda c &= \frac{\beta}{K_{eff}} \nu \Sigma_f \phi.
\end{split}
\end{equation} 
Here $\phi(\bm x,t)$ --- neutron flux  at point $\bm x$ and time $t$,
$v$ --- effective velocity of neutrons,
$D(\bm x)$ --- diffusion coefficient, 
$\Sigma_r(\bm x,t)$ --- removal cross-section,
$\Sigma_f(\bm x,t)$ --- generation cross-section,
$\beta$ --- effective fraction of delayed neutrons, 
$\lambda$ --- decay constant of sources of delayed neutrons.
System of equations (\ref{1}) is supplement with boundary condition and  initial condition.

The albedo-type conditions are set at the boundary of the area $\partial \Omega$:
\begin{equation}\label{2}
 D\frac{\partial \phi}{\partial n} + \gamma \phi = 0,
\end{equation}
where $n$ --- outer normal to the boundary $\partial \Omega$, $\gamma$ --- albedo constant.
Let's propose that in the initial time moment (at t = 0) the reactor is in the
critical state:
\begin{equation}\label{3}
 \phi(\bm x,0) = \phi^0(\bm x).
\end{equation} 
%Let's consider problem (\ref{1}) with boundary conditions (\ref{2}) and initial conditions

\section{Multiscale method}

\section{Test problem}


\section*{Acknowledgements}
This work was supported by the grant of the Russian Federation Government
(\#14.Y26.31.0013) and the Russian Science Foundation (\#19-71-00008).

\begin{thebibliography}{8}
\bibitem{Annals17}
Avvakumov,~A.~V., et al.: Spectral properties of dynamic processes in a nuclear reactor. Annals of Nuclear Energy. \textbf{99}, 68--79 (2017) 

\bibitem{Progress18}
Avvakumov A. V. et al.: State change modal method for numerical simulation of dynamic processes in a nuclear reactor. Progress in Nuclear Energy. \textbf{106}, 240--261 (2018)

\end{thebibliography}
\end{document}
