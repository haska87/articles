\documentclass{aip-cp}

\usepackage[numbers]{natbib}
\usepackage{rotating}
\usepackage{graphicx}

\usepackage{bm}

%\makeatletter
%\def\@fnsymbol#1{\ensuremath{\ifcase#1\or *\or \dagger\or **\or
%   \ddagger\or \mathsection\or \mathparagraph\or \|\or \dagger\dagger
%   \or \ddagger\ddagger \or\mathsection\mathsection
%   \or \mathparagraph\mathparagraph \or *{*}*\or
%   \dagger{\dagger}\dagger \or\ddagger{\ddagger}\ddagger\or
%   \mathsection{\mathsection}\mathsection
%   \or \mathparagraph{\mathparagraph}\mathparagraph \else\@ctrerr\fi}}
%\makeatother

% Document starts
\begin{document}

% Title portion
\title{Algorithm of Time Step Selection for Numerical Solution of Boundary Value Problem for Parabolic  Equations}

%\author[aff1]{Author's Name\corref{cor1}\noteref{note1,note2}}
%\eaddress[url]{http://www.aip.org}
%\author[aff2,aff3]{Author's Name\noteref{note2}}
%\eaddress{anotherauthor@thisaddress.yyy}

\author[aff1,aff2]{Petr Vabishchevich}
\author[aff1,aff2]{Aleksandr Vasilev\corref{cor1}}

\affil[aff1]{Nuclear Safety Institute of RAS, Moscow, Russia}
\affil[aff2]{North-Eastern Federal University, Yakutsk, Russia}
\corresp[cor1]{Corresponding author: haska87@gmail.com}

\maketitle


\begin{abstract}
We propose an algorithm allowing automatic time step evaluation when solving
the boundary value problem for parabolic equations.
The solution is obtained using complete stable implicit schemes, and the time step is evaluated using of the explicit scheme solution.
The time step evaluation formulas are derived using the estimation of the approximation error at next time step.
Calculation results obtained for model problem demonstrate reliability of the proposed algorithm for time step control. 
\end{abstract}

% Head 1
\section{INTRODUCTION}
For the second order parabolic equations unconditionally stable schemes are constructed on the basis of implicit approximations \cite{Matus}.
In computational practice two-layer schemes are mostly used, compared with three-layered and multilayered schemes which are not so often used.
The problem of the time step control is relatively well developed for the Cauchy problem solution of differential equations systems \cite{Ascher}. 
The basic approach is to use additional calculations at a new time step to estimate the approximate solution.
The time step is estimated using the theoretical asymptotic dependence of the accuracy on time step \cite{Vabishchevich}.

Additional calculations for estimating the error of the approximate solution can be carried out in different ways. The best-known strategy is connected with the solution of the problem on a separate time interval using the given step (the first solution) and with a step two times smaller (the second solution). The noted ways of selecting the time step are related to the class of a posteriori accuracy estimation methods.
The decision as to suitable the time step or the re-calculation is accepted only after the calculation is completed.

In this paper we consider in fact a priori choice of the time step in the approximate solution of boundary value problems for parabolic equations. The proposed algorithm allows a gain in CPU time with respect to the fine mesh calculation at the same calculation accuracy.

\section{PROBLEM DESCRIPTION}
Consider the second-order parabolic equation
\begin{equation}\label{1}
   \frac{\partial u}{\partial t} 
   - \sum_{\alpha =1}^{m}
   \frac{\partial }{\partial x_\alpha} 
   \left ( k({\bm x},t)  \frac{\partial u}{\partial x_\alpha} \right ) + s({\bm x},t) u = f({\bm x},t),
   \quad {\bm x}\in \Omega,
   \quad 0 < t \leq  T,
\end{equation}
where
$\underline{k} \leq k({\bm x}) \leq  \overline{k}, \ {\bm x} \in \Omega, \ \underline{k} > 0$.
The equation (\ref{1}) is supplemented by boundary  condition
\[
   u({\bm x},t) = g({\bm x},t),
   \quad {\bm x}\in \partial \Omega,
   \quad 0 < t \leq  T.
\]
and initial condition
\[
   u({\bm x},0) = u^0({\bm x}),
   \quad {\bm x}\in \Omega.
\]
After approximating over the space (by the method of finite differences or finite elements) we arrive at the Cauchy problem
\begin{equation}\label{2}
\frac{du}{dt} + A(t)u =f(t), \quad 0 < t \leq T,
\end{equation}
with initial condition $u(0) = u^0$. The problem is considered in a finite-dimensional Hilbert space.
We assume that $A(t) \geq 0$.
Let's use an irregular time grid
\[
t^0 = 0, \quad t^{n+1} = t^n + \tau^{n+1}, \quad 
n = 0, 1, ... , N-1, \quad t^N = T.
\]
For approximate solution the implicit scheme are used
\begin{equation}\label{3}
  \frac{y^{n+1} - y^{n}}{\tau^{n+1}} + A^{n+1} y^{n+1} = f^{n+1},
  \quad n = 0,1, ..., N-1, 
\end{equation}
and initial condition 
$
y^0 = u^0 .
$
For the approximate solution we can use the following estimate
\[
 \|y^{n+1}\| \leq \|y^{n}\| + \tau^{n+1} \|f^{n+1}\| .
\]
Then we can obtain a difference analogue of the estimate
\begin{equation}\label{4}
 \|y^{n+1}\| \leq \|u^{0}\| + \sum_{k=0}^{n} \tau^{k+1} \|f^{k+1}\|
\end{equation}
for problem (\ref{3}).
For the error of the approximate solution $z^n = y^n - u^n$ we have
\[
  \frac{z^{n+1} - z^{n}}{\tau^{n+1}} + A^{n+1} z^{n+1} = \psi^{n+1},
  \quad n = 0,1, ..., N-1,  \quad
 z^0 = 0.
\]
Here $\psi^{n+1}$ is approximation error
\begin{equation}\label{5}
 \psi^{n+1} = f^{n+1} -
 \frac{u^{n+1} - u^{n}}{\tau^{n+1}} - A^{n+1} u^{n+1} . 
\end{equation}
Similarly (\ref{4}) we have an estimate for the error
\begin{equation}\label{6}
  \|z^{n+1}\| \leq \sum_{k=0}^{n} \tau^{k+1} \|\psi^{k+1}\| .
\end{equation} 

Checking the error we can focus on the total error $\delta\tau^{n+1}$ in interval $t^n < t < t^{n+1}$. Then from (\ref{6}) we obtain
$\|z^{n+1}\| \leq \delta t^{n+1}.$
The error accumulates and increases linearly.
The solution is obtained using the unconditionally stable implicit scheme.
The major part of computing costs is related with this scheme.
The step control is performed using the explicit scheme solution.
The algorithm stability is not violated and is determined by the implicit scheme properties.
The error accumulation from the time layer $t^n$ to the new layer $t^{n+1}$  is defined as
\begin{equation}\label{7}
\|z^{n+1}\| \leq \|z^{n}\| + \tau^{n+1} \|\psi^{n+1}\| .
\end{equation}
Therefore, we have to control the local error $\psi^{n+1}$. Comparing $\psi^{n+1}$ with a given level of error $\delta$ we can control the time step choice. If $\psi^{n+1}$ is significantly larger (or smaller) then the $\delta$ - it means that the time step is too large (or small). Thereby
\begin{equation}\label{8}
  \tau^{n+1}: \ \|\psi^{n+1}\| \approx \delta .
\end{equation} 

Consider the basic algorithm to control the time step. We select the time step based on the analysis of the previous step solutions. The predicted time step is determined as following
\begin{equation}\label{9}
 \widetilde{\tau}^{n+1} = \gamma \tau^n , 
\end{equation}
where $\gamma$ is numerical parameter. The default value of $\gamma$ is 1.25 or 1.5. Using the explicit scheme we can obtain a solution $\widetilde{y}^{n+1}$ at time $\widetilde{t}^{n+1} = t^n + \widetilde{\tau}^{n+1}$. The calculation is performed only at single time step;  therefore the possible computational instability does not appear.
We estimate the approximation error using the calculated $\widetilde{y}^{n+1}$ by the implicit scheme. The $\tau^{n+1}$ is estimated by the proximity of the error norm to $\delta$. The solution at a new time step $t^{n+1}$ is calculated with a $\tau^{n+1}$  by the implicit scheme.

% Head 2
\section{CALCULATED FORMULAS}
We present the calculated formulas for the time step control. 
The predictive solution $\widetilde{y}^{n+1} $  is defined from
\[
  \frac{\widetilde{y}^{n+1} - y^{n}}{\widetilde{\tau}^{n+1}} + A^{n} y^{n} 
  = f^{n} .
\]
In accordance with (\ref{5}), the approximation error is calculated from the exact solution for two time moments: in our case, for $t^n$ and $\widetilde{t}^{n+1}$.
To estimate the error we take $u(t^n)$ instead of $y^n$. An exact solution at a new time step $u(\widetilde{t}^{n+1})$, is matched by an approximate solution $u(\widetilde{t}^{n+1})$, which is obtained by the explicit scheme. By virtue of this, we set
\begin{equation}\label{10}
 \widetilde{\psi}^{n+1}  = \widetilde{f}^{n+1} -
 \frac{\widetilde{y}^{n+1} - y^{n}}{\widetilde{\tau}^{n+1}} -
 \widetilde{A}^{n+1} \widetilde{y}^{n+1} . 
\end{equation} 
We match the approximation error $\widetilde{\psi}^{n+1}$ at the time step $\widetilde{\tau}^{n+1}$ and $\psi^{n+1}$ at the time step $\tau^{n+1}$.
Taking into account (\ref{8}), we set
\begin{equation}\label{11}
  \bar{\tau}^{n+1} = \gamma_{n+1} \tau^n,
  \quad \gamma_{n+1} = \frac{\delta}{\| \widetilde{\psi}^{n+1}\|}  \gamma.
\end{equation} 
The needed time step can not exceed the predicted time step, therefore
\[
\tau^{n+1} \leq \bar{\tau}_{n+1}, \quad \tau^{n+1} \leq \widetilde{\tau}_{n+1}.
\] 
We limit the allowable time step by the minimum step $\tau^0$:
\begin{equation}\label{12}
 \tau^{n+1} = \max \big \{\tau^0, \min \{\gamma_{n+1}, \gamma \} \tau^n \big \}. 
\end{equation}
Let's determine  the calculation formulas for the step selection algorithm in accordance with (\ref{10})-(\ref{12}):
\[
 \widetilde{\psi}^{n+1} = \widetilde{\tau}^{n+1} \left( \frac{\widetilde{f}^{n+1} - f^n}{\widetilde{\tau}^{n+1}}  - \frac{\widetilde{A}^{n+1} - A^n}{\widetilde{\tau}^{n+1}}  y^n - \widetilde{A}^{n+1} \frac{\widetilde{y}^{n+1} - y^n}{\widetilde{\tau}^{n+1}}  \right ) .
\] 
Thus, the approximation error has the first order for
time variable:
\[
 \widetilde{\psi}^{n+1} = \mathcal{O} (\widetilde{\tau}^{n+1}) .
\] 
By virtue of this, we set 
\begin{equation}\label{13}
 \|\widetilde{\psi}^{n+1} \| \leq \|\widetilde{f}^{n+1} - f^n  -
 (\widetilde{A}^{n+1} - A^n) y^n -
 \widetilde{A}^{n+1} (\widetilde{y}^{n+1} - y^n) \| .
\end{equation} 
Taking into account (\ref{13})  from (\ref{11}), we obtain the calculated formula for time step (\ref{12}), in which
\begin{equation}\label{14}
  \gamma_{n+1} = \frac{\delta}{ \|\widetilde{f}^{n+1} - f^n  -
  (\widetilde{A}^{n+1} - A^n) y^n  -
  \widetilde{A}^{n+1} (\widetilde{y}^{n+1} - y^n) \| } \gamma .
\end{equation} 
This formula (the denominator of the expression) clearly shows the corrective actions, which are associated with the change of the right-hand side (the first part), of the problem operator (the second part) and with the dynamics of the solution (the third part).

\section{TEST PROBLEM}
To illustrate the operability of the proposed algorithm, consider the boundary-value problem for a one-dimensional parabolic equation
\begin{equation}\label{15}
  \frac{\partial u}{\partial t} - \frac{\partial^2 u}{\partial x^2} + p(t) u = f(t),
  \quad 0 < x < 1,
  \quad 0 < t \leq  T ,  
\end{equation}
with boundary and initial conditions
\begin{equation}\label{16}
  u(0, t) = 0,
  \quad u(1,t) = 0,
  \quad 0 < t \leq  T , 
\end{equation}
\begin{equation}\label{17}
  u(x,0) = u^0(x),
  \quad  \quad 0 <  x <  1 .
\end{equation}
For an approximate solution of the problem (\ref{15})-(\ref{17}) we use the finite-difference approximation with respect to space. Define the uniform grid with step $h$
\[
  \bar{\omega} = \{ x \ | \ x = ih, \quad i = 0, 1, ..., M, \quad Mh = 1 \},  
\] 
where $\omega$ -- set of internal nodes, $\partial\omega$ -- set of boundary nodes ($\bar{\omega } = \omega \cup \partial \omega$).
% Acknowledgement
\section{ACKNOWLEDGMENTS}
This work are supported P.N. Vabishchevich by the Russian Foundation for Basic Research \#16-08-01215 and A.O. Vasilev by the Russian Foundation for Basic Research \#18-31-00315.

% References

\nocite{*}
\bibliographystyle{aipnum-cp}%
\bibliography{sample}%


\end{document}
