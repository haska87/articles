\documentclass[runningheads]{llncs}

\usepackage{graphicx}
\graphicspath{{./figs/}}

\usepackage{amsmath} 
\usepackage{bm}
\usepackage{amsbsy} 
\usepackage{amssymb}
\usepackage{cite}

\begin{document}
%
\title{Automatic Time Step Selection for Numerical Solution of Neutron Diffusion Problems}
%
%\titlerunning{Abbreviated paper title}
% If the paper title is too long for the running head, you can set
% an abbreviated paper title here
%
\author{A.V. Avvakumov\inst{1} \and
V.F. Strizhov\inst{2}\and
P.N. Vabishchevich\inst{2,3}\and
A.O. Vasilev\inst{3}}
%
\authorrunning{A.V. Avvakumov et al.}
% First names are abbreviated in the running head.
% If there are more than two authors, 'et al.' is used.
%
\institute{National Research Center \emph{Kurchatov Institute}, Moscow, Russia \and
Nuclear Safety Institute of RAS,  Moscow, Russia \and
North-Eastern Federal University, Yakutsk, Russia\\
\email{haska87@gmail.com}}
%
\maketitle              % typeset the header of the contribution
%
\begin{abstract}
Considering the approximate solution of the boundary value problems for nonstationary equations, the focus should be on the choice of time approximation schemes. 
For parabolic equations of the second order, unconditionally stable schemes are constructed on the basis of implicit approximations.
In computational practice two-layer schemes are mostly used, compared with three-layered and multilayered schemes which are not so often used.
We propose an algorithm allowing automatic time step evaluation when solving the boundary value problems	for parabolic equations.
The solution is obtained using guaranteed stable implicit schemes, and the step choice is performed with the use of the solution obtained by an explicit scheme.
Formulas for explicit calculation of the time step are derived using the estimation of the approximation error at new time step.
Calculation results obtained for some neutron diffusion problems demonstrate reliability of the proposed algorithm for time step choice. The algorithm takes into account the features of neutron diffusion problems, for instance, sharp changes in the solution or instability with respect to the initial data. When using the algorithm, you can expect a noticeable saving in calculation time compared with the fine mesh calculation, while maintaining the required accuracy of the calculation.

\keywords{First keyword  \and Second keyword \and Another keyword.}
\end{abstract}

\section{Introduction}
many much bukaffff 
many much bukaffff
many much bukaffff
many much bukaffff
many much bukaffff
many much bukaffff
many much bukaffff
many much bukaffff
many much bukaffff
many much bukaffff
many much bukaffff
many much bukaffff
many much bukaffff
many much bukaffff
many much bukaffff
many much bukaffff
many much bukaffff
many much bukaffff
many much bukaffff
many much bukaffff
many much bukaffff
many much bukaffff
many much bukaffff
many much bukaffff
many much bukaffff
many much bukaffff
many much bukaffff
many much bukaffff
many much bukaffff
many much bukaffff
many much bukaffff
many much bukaffff
many much bukaffff
many much bukaffff
many much bukaffff
many much bukaffff
many much bukaffff
many much bukaffff
many much bukaffff
many much bukaffff
many much bukaffff
many much bukaffff
many much bukaffff
many much bukaffff

\section{Problem description}
Consider the Cauchy problem for the linear equation
\begin{equation}\label{1}
\frac{du}{dt} + A(t)u =f(t), \quad 0 < t \leq T,
\end{equation}
with initial condition
\begin{equation}\label{2}
u(0) = u^0.
\end{equation}
The problem is considered in a finite-dimensional Hilbert space.
We assume that
\[
A(t) \geq 0.
\]
Introduce irregular time grid
\[
t^0 = 0, \quad t^{n+1} = t^n + \tau^{n+1}, \quad 
n = 0, 1, ... , N-1, \quad t^N = T.
\]
An approximate solution a purely implicit scheme are used
\begin{equation}\label{3}
\begin{split}
  \frac{y^{n+1} - y^{n}}{\tau^{n+1}} + A^{n+1} y^{n+1} = f^{n+1},
  \quad n = 0,1, ..., N-1, 
\end{split}
\end{equation}
and initial condition
\begin{equation}\label{4}
 y^0 = u^0.
\end{equation} 
For an approximate solution under constraints follows a layerwise estimate
\[
 \|y^{n+1}\| \leq \|y^{n}\| + \tau^{n+1} \|f^{n+1}\| .
\]
Then we obtain a difference estimate
\begin{equation}\label{5}
 \|y^{n+1}\| \leq \|u^{0}\| + \sum_{k=0}^{n} \tau^{k+1} \|f^{k+1}\|
\end{equation}
for problem (\ref{3}), (\ref{4}).
For the error of the approximate solution $z^n = y^n - u^n$, we have
\[
  \frac{z^{n+1} - z^{n}}{\tau^{n+1}} + A^{n+1} z^{n+1} = \psi^{n+1},
  \quad n = 0,1, ..., N-1,  
\] 
\[
 z^0 = 0.
\]
Here $\psi^{n+1}$ is approximation error:
\begin{equation}\label{6}
 \psi^{n+1} = f^{n+1} -
 \frac{u^{n+1} - u^{n}}{\tau^{n+1}} - A^{n+1} u^{n+1} . 
\end{equation}
Similarly (\ref{5}) we have an estimate for the error
\begin{equation}\label{7}
  \|z^{n+1}\| \leq \sum_{k=0}^{n} \tau^{k+1} \|\psi^{k+1}\| .
\end{equation} 
Checking the error we can focus on the total error $\delta\tau^{n+1}$ in interval $t^n < t < t^{n+1}$. Then from (\ref{7}) we obtain
\[
 \|z^{n+1}\| \leq \delta t^{n+1}.
\]
The error accumulates and increases linearly.

The solution is obtained using guaranteed stable implicit schemes.
This scheme includes main computing costs.
The step choice is performed with the use of the solution obtained by an explicit scheme.
Formulas for explicit calculation of the time step are derived using the estimation of the approximation error at new time step.
The stability of our algorithm is not violated. 
It is determined by the properties of the implicit scheme.

The accumulation of errors in the transition from the time layer $t^n$ to the new layer $t^{n+1}$  is defined as
\begin{equation}
\|z^{n+1}\| \leq \|z^{n}\| + \tau^{n+1} \|\psi^{n+1}\| .
\end{equation}
Therefore, we must control the local error $\psi^{n+1}$. Comparing $\psi^{n+1}$ with a given level of error $\delta$ we can judge the quality of the time step choice. If $\psi^{n+1}$ is significantly larger (smaller) then the $\delta$ that the time step is taken too large (small). Thereby
\begin{equation}\label{9}
  \tau^{n+1}: \ \|\psi^{n+1}\| \approx \delta .
\end{equation} 

Consider the main algorithm for choosing the time step. We select the predicted step in time based on the analysis of the solution at the previous steps in time. The predicted time step is determined as following
\begin{equation}
 \widetilde{\tau}^{n+1} = \gamma \tau^n , 
\end{equation}
where $\gamma$ is numeric parameter. The default parameter of $\gamma$ is 1.25 or 1.5. Using an explicit scheme we find a solution $\widetilde{y}^{n+1}$ at time $\widetilde{t}^{n+1} = t^n + \widetilde{\tau}^{n+1}$. The calculation is done only on one step in time therefore the possible computational instability does not appear.
We estimate the error of approximation by the found $\widetilde{y}^{n+1}$ using the implicit scheme. The $\tau^{n+1}$ is estimated by the proximity of the error norm to $\delta$. The solution at a new time step $t^{n+1}$ is calculated with a $\tau^{n+1}$  by a purely implicit scheme.

\section{Calculated formulas}

We present the calculated formulas for the choice of the time step for the neutron diffusion equation.
\subsection{Neutron diffusion equation}

Let's consider modelling neutron flux in a one group diffusion approximation with one group delayed neutron sources. Neutron flux dynamics is considered within a bounded 2D domain  $\Omega$ ($\bm x = \{x_1, x_2\} \in \Omega$) with a convex boundary $\partial \Omega$.
 \begin{equation}
\begin{split}
 \frac{1}{v} \frac{\partial \phi}{\partial t} -  \nabla \cdot D \nabla \phi + \Sigma_{a} \phi &=   \ (1-\beta) \nu \Sigma_{f} \phi + \lambda c, \\
\frac{\partial c}{\partial t} + \lambda c &= \beta \nu \Sigma_{f} \phi.
\end{split}
\end{equation} 
Here $\phi(\bm x,t)$ is the neutron flux at point $\bm x$ and time $t$,
$v$ is the effective velocity of neutrons,
$D(\bm x)$ is the diffusion coefficient, 
$\Sigma_{a}(\bm x,t)$ is the absorption cross-section,
$\beta$ is the effective fraction of delayed neutrons, 
$\nu\Sigma_{fg}(\bm x,t)$ is the generation cross-section,
$c$ is density of source of delayed neutrons, 
$\lambda$  is decay constant of source of delayed neutrons.

The conditions so-called albedo-type are set at the boundary $\partial \Omega$:
\begin{equation}
 D\frac{\partial \phi}{\partial n} + \gamma \phi = 0.
\end{equation}
where $n$ is the outer normal to the boundary $\partial \Omega$.
Let's consider problem (\ref{1}) with boundary conditions (\ref{2}) and initial conditions:
\begin{equation}
 \phi(\bm x,0) = \phi^0(\bm x),  \quad  
 c(\bm x, 0) = c^0(\bm x).
\end{equation} 

%For convenience rewrite the boundary problem (\ref{1}), (\ref{2}), (\ref{3}) in operator notation
%\begin{equation}
%V \frac{d\bm{\phi}}{dt}  + A\bm{\phi} = 0,
%\end{equation}
%where $\bm{\phi} = \{\phi, c\}$ and
%\[
%\begin{split}
% V = \begin{pmatrix}
% \frac{1}{v}\\
% 1
% \end{pmatrix},
% \quad
%A &= \begin{pmatrix}
%  - \nabla \cdot D \nabla  + \Sigma_{a}  - (1-\beta) \nu \Sigma_{f} - \lambda & 0  \\
%  0  & \lambda - \beta\nu\Sigma_f   \\
% \end{pmatrix}.
% \end{split}
%\]
%The Cauchy problem is formulated for equation (\ref{4}) when
%\begin{equation}
%\bm{\phi}(0) = \bm\phi^0.
%\end{equation}
%Space discretization is performed using finite element method with standard Lagrange finite elements.

\subsection{Time step estimate}
In our case, the approximation error $\bm{\psi^{n+1}} = \{\psi^{n+1}_1, \psi^{n+1}_2\}$ is
\begin{equation}\label{14}
\begin{split}
\psi^{n+1}_1  & =  \frac{1}{v} \frac{\phi^{n+1}-\phi^n}{\tau^{n+1}} - \nabla \cdot D^{n+1} \nabla \phi^{n+1} + \Sigma_{a}^{n+1} \phi^{n+1} \\
 & - (1-\beta) \nu \Sigma^{n+1}_{f} \phi^{n+1} - \lambda  c^{n+1}, \\
\psi^{n+1}_2  & =  \frac{ c^{n+1}-c^n}{\tau^{n+1}} + \lambda c^{n+1} - \beta \nu\Sigma_{f}^{n+1} \phi^{n+1},
\end{split}
\end{equation} 
where $\bm{\phi^{n+1}} = \{ \phi^{n+1}, c^{n+1} \}$ is exact solution.
The predictive solution is defined as
\begin{equation}\label{15}
\begin{split}
\frac{1}{v} \frac{\widetilde\varphi^{n+1}-\varphi^n}{\widetilde\tau^{n+1}} - \nabla \cdot D^n \nabla \varphi^n + \Sigma^n_{a} \varphi^n & = (1-\beta) \nu \Sigma^n_{f} \varphi^n + \lambda s^n, \\
\frac{\widetilde s^{n+1}-s^n}{\widetilde\tau^{n+1}} + \lambda s^n &= \beta \nu\Sigma^n_{f} \varphi^n.
\end{split}
\end{equation} 


\section{Computational results}
				


% ---- Bibliography ----
%
% BibTeX users should specify bibliography style 'splncs04'.
% References will then be sorted and formatted in the correct style.
%
% \bibliographystyle{splncs04}
% \bibliography{mybibliography}
%
\begin{thebibliography}{8}
\bibitem{ref_article1}
Author, F.: Article title. Journal \textbf{2}(5), 99--110 (2016)

\bibitem{ref_lncs1}
Author, F., Author, S.: Title of a proceedings paper. In: Editor,
F., Editor, S. (eds.) CONFERENCE 2016, LNCS, vol. 9999, pp. 1--13.
Springer, Heidelberg (2016). \doi{10.10007/1234567890}

\bibitem{ref_book1}
Author, F., Author, S., Author, T.: Book title. 2nd edn. Publisher,
Location (1999)

\bibitem{ref_proc1}
Author, A.-B.: Contribution title. In: 9th International Proceedings
on Proceedings, pp. 1--2. Publisher, Location (2010)

\bibitem{ref_url1}
LNCS Homepage, \url{http://www.springer.com/lncs}. Last accessed 4
Oct 2017
\end{thebibliography}
\end{document}
